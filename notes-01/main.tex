\documentclass{../doc}

\begin{document}
  \header{Presentation Notes -- Week 1}{
    Private Reading: Quantum Computing \\
    \today \\
    Oberlin College \\
    Iago B. Mendes
  }

  \section{Classical bits (Cbits)}
    \begin{itemize}
      \item 2 possible states: $\ket{0}$ or $\ket{1}$
      \item Reversible operations
        \begin{itemize}
          \item Definition: initial states can be retrieved with the knowledge of the final states and the operation
          \item Example of irreversible operations: ERASE
            \begin{align}
              \ket{0} &\to \ket{0} \\
              \ket{1} &\to \ket{0}
            \end{align}
          \item For $n$ Cbits, there are $(2^n)!$ reversible operations \\
            $\Rightarrow$ mapping each state to another one (permutation of states)
          \item 1-Cbit operators:
            \begin{itemize}
              \item Identity
                \begin{align}
                  \hat 1 \ket{1} &= \ket{1} \\
                  \hat 1 \ket{0} &= \ket{0} \\
                  \hat 1 &= \begin{pmatrix}
                    1 & 0 \\
                    0 & 1
                  \end{pmatrix}
                \end{align}
              \item NOT
                \begin{align}
                  \hat X \ket{1} &= \ket{0} \\
                  \hat X \ket{0} &= \ket{1} \\
                  \hat X &= \begin{pmatrix}
                    0 & 1 \\
                    1 & 0
                  \end{pmatrix}
                \end{align}
            \end{itemize}
          \item Example of 2-Cbit operator:
            \begin{itemize}
              \item cNOT ($\hat C_{ij}$): ``controlled-NOT'' \\
                $i \to$ ``control Cbit'' (determines when to apply NOT) \\
                $j \to$ ``target Cbit'' (receives NOT)
                \begin{align}
                  \hat C_{01} \ket{10} &= \ket{11} \\
                  \hat C_{10} \ket{10} &= \ket{10} \\
                  \hat C_{00} \ket{10} &= \ket{00} \\
                  &\vdots \notag\\
                  \hat C_{01} &= \begin{pmatrix}
                    1 & 0 & 0 & 0 \\
                    0 & 1 & 0 & 0 \\
                    0 & 0 & 0 & 1 \\
                    0 & 0 & 1 & 0
                  \end{pmatrix} \\
                  &\vdots \notag
                \end{align}
              
              % \item SWAP ($\hat S_{ij}$): swaps the $i$th and $j$th states
              %   \begin{align}
              %     \hat S_{10} \ket{xy} &= \ket{yx} \\
              %     &\vdots \notag\\
              %     \hat S_{10} = \hat S_{01} &= \begin{pmatrix}
              %       1 & 0 & 0 & 0 \\
              %       0 & 0 & 1 & 0 \\
              %       0 & 1 & 0 & 0 \\
              %       0 & 0 & 0 & 1
              %     \end{pmatrix} \\
              %     &\vdots \notag
              %   \end{align}
              %   \begin{itemize}
              %     \item Can be constructed out of three cNOT's
              %       \begin{gather}
              %         \hat S_{ij} = \hat C_{ij} \hat C_{ji} \hat C_{ij}
              %       \end{gather}
              %   \end{itemize}
            \end{itemize}
        \end{itemize}
      \item ``For a reversible classical computer, at least one 3-Cbit gate is needed to build up general logical operations'' \\
        $\Rightarrow$ Toffoli gate
    \end{itemize}

  \section{Quantum bits (Qbits)}
    \begin{itemize}
      \item Infinite possible states:
        \begin{equation}
          \ket{\psi} = \alpha_0 \ket{0} + \alpha_1 \ket{1}
        \end{equation}
        $\alpha_0, \alpha_1 \in \mathbb{C}$
      \item $n$-Qbit system:
        \begin{equation}
          \ket{\Psi} = \sum_{x=0}^{2^n-1} \alpha_x \ket{x}_n
        \end{equation}
      \item Reversible operations: any unitary transformations! \\
        $\Rightarrow$ maintain normalized amplitudes
        \begin{align}
          \hat u \hat u^\dagger &= \hat 1 \quad\text{(1-Qbit)}\\
          \hat U \hat U^\dagger &= \hat 1 \quad\text{($n$-Qbit)}
        \end{align}
        \begin{itemize}
          \item Hadamard ($\hat H$): definite state $\to$ superposition
            \begin{align}
              \hat H \ket{0} &= \frac{1}{\sqrt{2}} (\ket{0} + \ket{1}) \\
              \hat H \ket{1} &= \frac{1}{\sqrt{2}} (\ket{0} - \ket{1}) \\
              \hat H &= \frac{1}{\sqrt{2}} \begin{pmatrix}
                1 & 1 \\
                1 & -1
              \end{pmatrix}
            \end{align}
        \end{itemize}
      \item For general logical operations, we only need 1-Qbit and 2-Qbit gates \\
        $\Rightarrow$ can construct a Toffoli gate (shown in the next chapter)
    \end{itemize}
  
  \section{Measurement}
    \begin{itemize}
      \item Only irreversible operation of quantum computers \\
        $\Rightarrow$ collapse of the state
      \item ``Born rule'': probability of an outcome
        \begin{equation}
          p(x) = |\alpha_x|^2
        \end{equation}
      \item ``Generalized Born rule'': measurement of only one Qbit
        \begin{equation}
          \ket{\Psi}_{n+1} = \alpha_0 \ket{0} \ket{\Phi_0}_n + \alpha_1 \ket{1} \ket{\Phi_1}_n
        \end{equation}
        If applied multiple times, gives the usual Born probability \\
        $\Rightarrow$ can use only 1-Qbit measurement gates
    \end{itemize}

\end{document}
