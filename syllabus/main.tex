\documentclass{../doc}

\begin{document}
  \header
    {Private Reading: Quantum Computing}
    {Phys 995F, 4 Credits \\ Spring 2025}

  \begin{multicols}{2}
    \section*{Instructor}
      \begin{itemize}
        \item Name: Robert Owen
        \item E-mail: \href{mailto:rowen@oberlin.edu}{rowen@oberlin.edu}
        \item Department: Physics and Astronomy
      \end{itemize}
    \columnbreak
    \section*{Student}
      \begin{itemize}
        \item Name: Iago Braz Mendes
        \item E-mail: \href{mailto:ibrazmen@oberlin.edu}{ibrazmen@oberlin.edu}
        \item T number: T01362926
    \end{itemize}
  \end{multicols}

  \begin{multicols}{2}
    \section*{Meetings}
      \begin{itemize}
        \item Time: Fridays, 12:30 pm
        \item Location: Wright 103
      \end{itemize}
    \columnbreak
    \section*{Textbook}
      ``Quantum Computer Science: An Introduction'' by N.~David Mermin
  \end{multicols}

  \section*{Course Description}
    This private reading consists of an introduction to quantum computation, developing the basic elements of the field from a computer-science perspective. Since quantum computers are finite-state systems, we can ignore much of the complexity of quantum mechanics in dealing with continuous systems, focusing on abstract models that allow us to solve computational problems.
  
  \section*{Learning Goals}
    \begin{itemize}
      \item To become familiar with how one can use the theory of quantum mechanics to perform computations;
      \item To see famous applications of quantum computing and examples of speed-ups; and
      \item To get experience with the implementation and execution of quantum-computing protocols.
    \end{itemize}

  \section*{Assignments}
    \begin{itemize}
      \item {\bf Weekly Presentations:} At every meeting, Iago will present on the topic of the week's reading. These presentations will be between 30 minutes and 1 hour in length. In preparation, Iago will write presentation notes that will serve as a summary for future reference.
      \item {\bf Coding Projects}: After Spring Break, Iago will focus on the parts of the textbook about protocols that use just a few Qbits. At that time, Iago's preparation will also involve implementing these protocols in code and potentially running them in quantum computers from IBM's free-access platform.
    \end{itemize}

  \section*{Evaluation}
    The final grade will be determined on a Pass / No Pass basis. The completion of 80\% of the assignments listed above will result in a passing grade.

  \section*{Tentative Schedule}
    \begin{center}
      \setlength{\tabcolsep}{0.5cm}
      \def\arraystretch{2}
      \begin{tabular}{cll}
        \Centering{\bf Week} & \Centering{\bf Topic} & \Centering{\bf Reading} \\
        \hline
        \hline
        1 (02/04) & Course overview & -- \\
        \hline
        2 (02/14) & Cbits and Qbits & Ch.~1 \\
        \hline
        3 (02/21) & General features and examples & Ch.~2 \\
        \hline
        4 (02/28) & Breaking RSA encryption & Ch.~3 \\
        \hline
        5 (03/07) & Searching with a quantum computer & Ch.~4 \\
        \hline
        6 (03/14) & Quantum error correction & Ch.~5 \\
        \hline
        -- (03/21) & {\em Away for a conference} & -- \\
        % \hline
        -- (03/28) & {\em Spring break} & -- \\
        \hline
        7 (04/04) & Project: Bells state & Sec.~6.1 \\
        \hline
        8 (04/11) & Project: Quantum cryptography & Sec.~6.2 \\
        \hline
        9 (04/18) & Project: Bit commitment & Sec.~6.3 \\
        \hline
        10 (04/25) & Project: Quantum dense coding & Sec.~6.4 \\
        \hline
        11 (05/02) & Project: Teleportation & Sec.~6.5 \\
        \hline
        12 (05/09) & Project: The GHZ puzzle & Sec.~6.6 \\
        \hline
        \hline
      \end{tabular}
    \end{center}

\end{document}
